\section{Instead of a conclusion}

The main driving force leading to this article was the following idea: 
using Haskell, describe -- in a clear and modular way --
the main ingredients of a minimalistic supercompiler, and to show how they fit together.
As a result we arrived at the following definition:
\begin{lstlisting}
supercompile :: Task -> Task
supercompile (e, p) =
	residuate $ simplify $ foldTree $
		buildFTree (addPropagation $ driveMachine p) e
\end{lstlisting}
The text of the article is, in fact, just an attempt to describe each
of the parts of this definition, and to show what effects they can
achieve together.

We hope the reader will take some time to study SC~Mini's sources,
where comments describe some interesting technical details of
the implementation.

\subsection{Where can we go from here?}

If you become more deeply interested in supercompilation, here a short
list of possible next steps:
%\marginpar{DK: Some English replacement for SA videos?}

\begin{enumerate}
%  \item Видеозаписи лекций Сергея Абрамова по метавычислениям. (TODO: выложить на Vimeo!!).
%% status of this todo?
%  Обязательно посмотрите первую лекцию~--- Сергей Михайлович великолепный рассказчик!
  \item The articles \cite{Sorensen1998Introduction,Sorensen1996Positive} are 
  still very good and readable introductions to the standard techniques used in supercompilation,
  including some not covered here, such as using homeomorphic embedding as a whistle,
  using most specific generalization as a generalization algorithm.
%  \marginpar{DK: something else in English?}
%  \item Блог Сергея Романенко, посвященный суперкомпиляции \url{http://metacomputation-ru.blogspot.com/}
  \item the Google group ``Supercompilation and Related Techniques'' contains the most 
  up-to-date discussions and announcements concerning supercompilation.
  \end{enumerate}

%\subsection{An invitation for collaboration}
%
%TODO: DK: Do we leave this? If yes, how to rewrite?

%История суперкомпиляции напоминает раскачивание маятника - от простого к сложному, от сложного к
%простому.
%
%Первый период~--- суперкомпиляция РЕФАЛа~--- развитие от простого к сложному: сделать хоть
%какой-нибудь (сложный) суперкомпилятор, лишь бы он автоматически работал и выдавал разумные
%результаты. Второй период~--- суперкомпиляция функциональных языков первого порядка~--- осмысление
%полученных результатов, их схематизация и отчуждение. Нынешний период~--- это опять движение от
%простого (осмысленного) к сложному (пока еще непонятному). Это что касается практики.
%
%С другой стороны, в области, близкой к суперкомпиляции~--- частичных вычислениях~--- очень важные
%результаты были получены при экспериментах на совсем игрушечных, не имеющих практической ценности,
%частичных вычислителях для игрушечных языков.
%
%Одно из крайне интересных для меня направлений для исследований~--- проведение различного рода
%экспериментов на простых и понятных суперкомпиляторах. Проблема, что таких суперкомпиляторов
%пока нет. Данная статья~--- это попытка движения в этом направлении. Предложенный суперкомпилятор
%SC~Mini невелик, но некоторые части в нем написаны не очень элегантным способом. Буду рад
%конструктивным предложениям и замечаниям по его улучшению.

% Предложения и замечания по данной статье также приветствуются.

% Опубликовать на гитхабе и дать ссылку с четким позывом "форкайте и исправляйте" и указанием, что
% на гитхабе форкать - одно удовольствие.
% Дать какие-либо метрики кода, чтобы люди поняли, что покопаться в нем - более чем обозримая задача.

